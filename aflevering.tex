\documentclass[a4paper, oneside]{scrartcl}
\usepackage[danish]{babel}
\usepackage[utf8]{inputenc}
\usepackage{enumerate}

\title{Take-Home Eksamen}
\subtitle{DM500: Studieintroduktion for datalogi}
\date{Efterår 2021}
\author{Gruppe 3: Andreas Pedersen, Sebastian Brendel, Søren Johansen}

\begin{document}
\maketitle  
\section*{Reeksamen 2015 DM549 Opgave 1}
\begin{enumerate}[a)]
\item Mængden A er lig \[A = \{2, 4, 6, 8\}\]

\item Mængden b er lig \[B = \{5, 8, 11, 14\}\]

\item Fællesmængden af A og B er lig \[A \cap B = \{8\}\]

\item Foreningsmængden af A og B er lig \[A \cup B = \{2, 4, 5, 6, 8, 11, 14\}\]

\item A fraregnet B er lig \[A - B = \{2, 4, 6\}\]

\item Komplementet til A = \[\overline{A} = U - A = \{1, 3, 5, 7, 9, 10, 11,12, 13, 14, 15\}\]
\end{enumerate}

\section*{Eksamen januar 2012 Opgave 1}

\begin{enumerate}[a)]
    \item Funktionen \(f\) er ikke en bijektion, da den ikke injektiv, på grund af andengradsleddet, der gør at den har flere \(x\) til samme funktionsværdi.
    \item \(f\) f kan ikke inverteres fordi den ikke er bijektiv, altså en eventuel invers funktion vil have to værdier til ét input, hvorfor det ikke er en funktion, og dermed har f ikke en invers funktion.
    \item Summen af funktionerne \( f\) og \(g\) \[f + g = x^2 + x + 1 + 2x -2 = 22 -x -1\]
    \item Den sammensatte funktion er den ene funktion på den anden funktion: \[ g \circ f = g(f(x) = 2(x^2 + x + 1) - 2 = 2x^2 + 2x \]

\end{enumerate}

\end{document}
